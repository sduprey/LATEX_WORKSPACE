\subsection{Partie r\'eguli\`ere}\label{equationintegralesectionsub105}
\noindent Les int\'egrales \'el\'ementaires que l'on est amen\'ees \`a estimer num\'eriquement lors de l'interaction entre les degr\'es de libert\'e de l'\'el\'ement ligne $L$ et de l'\'el\'ement colonne $K$ s'explicitent toutes en fonction de la quantit\'e suivante :
\BEQ{deuxiemenotationsimplificatrice}
G_{m,p}^{ij}=\int_\Gamma{\rm d} \sigma \int_{\Gamma'}{\rm
d} \sigma '\oint_0^{\pi}{\rm d}\theta \ \  \sigma ^{i-1} \sigma '^{j-1}\frac{e^{ik
R(r( \sigma ),r'( \sigma '),z( \sigma ),z'( \sigma '),{\rm cos}\theta)}}{ R^p
(r( \sigma ),r'( \sigma '),z( \sigma ),z'( \sigma '),{\rm cos}(\theta))}{\rm cos}(m\theta)
\EEQ
Ces int\'egrales ne sont pas toujours d\'efinies : elles sont singuli\`eres dans le cas de l'autor\'eaction, o\`u l'\'el\'ement $L$ et l'\'el\'ement $K$ co�ncident. Lorsque les deux \'el\'ements $K$ et $L$ sont \'eloign\'es (\`a un seuil de tol\'erance que l'on a fix\'e), ces int\'egrales ne sont plus singuli\`eres et peuvent \^etre approxim\'ees num\'eriquement par une m\'ethode de quadrature.
\subsubsection{Quadrature}
\noindent \\ Toute int\'egrale du type $I=\int_a^bf(x) {\rm d}x$ est ramen\'ee par un changement de variable affine sur l'intervalle ]-1,1[ :
\begin{displaymath}
I=\int_a^bf(x) {\rm d}x=\frac{b-a}{2}\int_{-1}^1 f(\frac{b-a}{2}u+\frac{b+a}{2}) {\rm d}u = \frac{b-a}{2}\int_{-1}^1 \phi(u) {\rm d}u
\end{displaymath}
Une formule de quadrature sur ]-1,1[ est la donn\'ee de $n$ points $u_i$ de ]-1,1[ et de $n$ poids $w_i$, permettant d'approximer l'int\'egrale sur ]-1,1[ : 
\begin{displaymath}
\int_{-1}^1 \phi(u) {\rm d}u=\sum_{i=1}^nw_i \phi(x_i)
\end{displaymath}
Les int\'egrales sur un intervalle ]a,b[ quelconque se d\'eduisent de cette quadrature sur ]-1,1[ :
\begin{displaymath}
I=\frac{b-a}{2}\sum_{i=1}^nw_i \phi(u_i)=\frac{b-a}{2}\sum_{i=1}^nw_i f(\frac{b-a}{2}u_i+\frac{b+a}{2})
\end{displaymath}
On choisit ici une quadrature de Gauss. Les points d'interpolation sont les z\'eros des polyn\^omes de Legendre (\ref{definitionanalytiquepolynomedelegendreassociee}), tous r\'eels et strictement contenus dans $]-1,1[$. Ces z\'eros se r\'epartissent sym\'etriquement par rapport \`a l'origine (le polyn\^ome de Legendre a m\^eme parit\'e que l'entier $n$ et pour un nombre $n$ de points impairs, le milieu de l'intervalle appartient \`a� la quadrature) et les extr\'emit\'es de l'intervalle n'appartiennent pas aux points d'interpolation. Pour $n$ points d'interpolation, cette formule est exacte pour les polyn\^omes de degr\'e $2n-1$.
\subsubsection{M\'ethodologie Num\'erique}
\begin{center}
\epsfig{file=nacellenumtrois.eps,height=5.5cm,width=15cm}
Trois boucles successives 
\end{center}
\paragraph{Premi\`ere \'etape}
\noindent
\\
Pour chaque point de Gauss $ \dsp pG_K$ de l'\'el\'ement colonne $ \dsp K$ de   $\dsp \Gamma$ de coordonn\'ees $ \dsp (r'_{pG_K},z'_{pG_K})$ et pour chaque point de Gauss angulaire $ \dsp pG_\theta$ d'angle $ \dsp \theta_{pG_\theta}$, on calcule l'int\'egration sur l'\'el\'ement ligne $ \dsp L$ :
\BEQ{bouclepointdegaussdeuxiemeelement}
 \dsp g_L(p,i,pG_K,\theta_{pG_\theta})=\sum_{pG_L}w_L(pG_L)(t_{pG_L})^{i-1}\frac{e^{ikR}}{R^p}\frac{d_L}{2},
\EEQ 
o\`u $ \dsp t_{pG_L}$ (resp. $ \dsp w_L(pG_L)$) repr\'esente l'abscisse curviligne (resp. le poids de Gauss) du point de Gauss $ \dsp pG_L$ de l'\'el\'ement ligne $ \dsp L$ et $ \dsp R^2=(z_{pG_L}-z'_{pG_K})^2+r_{pG_L}^2+r_{pG_K}'^2-2r_{pG_L}r_{pG_K}'{\rm
cos}(\theta_{pG_\theta})$  d\'esigne la distance entre le point de Gauss $ \dsp pG_K$ en cours pour une surface $ \dsp \Gamma '$ tourn\'ee de $ \dsp \theta_{pG_\theta}$ au le point de Gauss $ \dsp pG_L$.  
\paragraph{Deuxi\`eme \'etape}
\noindent \\
On int\'egre sur le secteur angulaire :
\BEQ{bouclepointdegaussangulaire}
 \dsp  g_{K}(i,m,p)=\sum_{pG_\theta}w_\theta(pG_\theta){\rm cos}(m
 \theta) g_L(p,i,pG_K,\theta_{pG_\theta})\frac{\pi}{2}
\EEQ
\paragraph{Troisi\`eme \'etape}
\noindent \\ On int\'egre sur l'\'el\'ement colonne $K$:
\BEQ{gausscompletregulier}
 \dsp G_{m,p}^{ij}=\sum_{pG_K}(t'_{pG_K})^{j-1}w_K(pG_K)g_{K}(i,m,p)\frac{d_K}{2}
\EEQ
En rassemblant les trois sommations successives : 
\BEQ{gausscompletregulier}
 \dsp  G_{m,p}^{ij}=\sum_{pG_K}w_K(pG_K)(t'_{pG_K})^{j-1}\sum_{pG_\theta}w_\theta(pG_\theta){\rm
cos}(m
 \theta)\sum_{pG_L}(t_{pG_L})^{i-1}w_L(pG_L)\frac{e^{ikR}}{R^p}\frac{d_Ld_K\pi}{8}
\EEQ
\subsection{Int\'egrales singuli\`eres}\label{equationintegralesectionsub106}
\noindent Lorsque les \'el\'ements $K$ et $L$ sont proches \`a un seuil $\epsilon $ fix\'e, la singularit\'e de la fonction de Green ne permet pas une int\'egration pr\'ecise par une formule de quadrature. Il faut extraire la partie singuli\`ere non oscillante de l'int\'egrale, que l'on int\`egre maintenant analytiquement. La partie r\'eguli\`ere est toujours \'evalu\'ee \`a l'aide d'une formule de quadrature.
Les parties singuli\`eres et r\'eguli\`eres s'explicitent pour les quatres op\'erateurs int\'egraux :
\newline
Le d\'eveloppement limit\'e $\dsp \frac{e^{ikR}}{R}=\frac{1}{R}+ik+ \dsp o_{R\rightarrow 0}(1)$ du noyau de Green donne la partie r\'eguli\`ere $\dsp \frac{e^{ikR}}{R}-\frac{1}{R}$ pour les op\'erateurs $S$ et $N$.
\newline
Le d\'eveloppement limit\'e $\dsp \left(ik-\frac{1}{R}\right)\frac{e^{ikR}}{R^2}=-\frac{1}{R^3}-\frac{k^2}{2R}-\frac{ik^3}{3}+ \dsp o_{R\rightarrow 0}(1)$ donne la partie r\'eguli\`ere $\dsp \left(ik-\frac{1}{R}\right)\frac{e^{ikR}}{R^2}+\frac{1}{R^3}+\frac{k^2}{2R}$ pour les op\'erateurs $D$ et $D^*$.
\newline Les int\'egrales suivantes des parties singuli\`eres :
$$\dsp \int_{S_{1L}}^{S_{2L}}( \sigma  )^{i-1}\frac{1}{R}{\rm d} \sigma , \ \dsp \int_{S_{1L}}^{S_{2L}}( \sigma )^{i-1}\frac{1}{R^2}{\rm d} \sigma \text{ et }\dsp \int_{S_{1L}}^{S_{2L}}( \sigma )^{i-1}\frac{1}{R^3}{\rm d} \sigma, \text{ pour } i\in \lbrace 1,2,3 \rbrace,$$ sont estim\'ees analytiquement.