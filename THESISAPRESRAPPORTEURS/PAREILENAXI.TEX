\subsubsection{Formulation axisym\'etrique}\label{pareilenaxisubsectionsub1}
\noindent Nous \'enon\c cons ci-dessous l'analogue axisym\'etrique du paragraphe pr\'ec\'edent (\ref{unicitesommerfeldsubsection1}).
\newline Les domaines consid\'er\'es sont maintenant les domaines de coupe transverses bidimensionnels, que l'on note par abus de notation comme les domaines tridimensionnels cart\'esiens.
\newline Les domaines et les variables physiques transform\'es sont toujours not\'es par abus de notation comme les anciens.
\newline
$\bullet$ Le mode azimutal $m$ du potentiel acoustique total transform\'e $\phi_{am}$ dans l'espace de Lorentz cylindrique v\'erifie l'\'equation aux d\'eriv\'ees partielles volumique :\BEQ{lorentztoutpartoutpartoupartou2r}
\left\{
\begin{array}{l}
\dsp\frac{1}{r}\frac{\partial}{\partial r}\left(r\rho_0\frac{\partial\phi_{am}}{\partial r}\right) +\frac{1}{1-M_\infty^2}\frac{\partial}{\partial z} \left(\rho_0\frac{\partial \phi_{am}}{\partial z}\right)\\
\dsp -\frac{1}{\sqrt{1-M_\infty^2}}\frac{\partial}{\partial r}\left(\rho_0M_{0r}M_{0z}\frac{\partial \phi_{am}}{\partial z}\right)-\frac{1}{\sqrt{1-M_\infty^2}}\frac{\partial}{\partial z}\left(\rho_0 M_{0z} M_{0r}\frac{\partial \phi_{am}}{\partial r}\right) \\
\dsp-\frac{1}{1-M_\infty^2}\frac{\partial}{\partial z}\left(\rho_0 M_{0z}^2\frac{\partial \phi_{am}}{\partial z}\right)-\frac{\partial}{\partial r}\left(\rho_0 M_{0r}^2\frac{\partial \phi_{am}}{\partial r}\right)\\ 
\dsp +\rho_0\left( \frac{ik_\infty M_\infty}{\left(1-M_\infty\right)^{\frac{3}{2}}}\left(1-M_{0z}^2\right)+\frac{ik_0 M_{0z}}{\sqrt{1-M_\infty^2}}-\frac{M_{0r}M_{0z}}{r \sqrt{1-M_\infty^2}}\right)\frac{\partial \phi_{am}}{\partial z}\\
\dsp +\rho_0\left(-\frac{M_{0r}^2}{r}+ik_0M_{0r}-\frac{ik_\infty M_\infty M_{0r}M_{0z}}{1-M_\infty^2}\right)\frac{\partial \phi_{am}}{\partial r}\\
\dsp+\frac{\partial}{\partial r}\left(i k_0\rho_0 M_{0r}\phi_{am}\right)+\frac{ik_\infty M_\infty}{\left(1-M_\infty^2\right)^{\frac{3}{2}}}\frac{\partial \left(\rho_0\phi_{am}\right)}{\partial z}-\frac{ik_\infty M_\infty}{1-M_\infty^2}\frac{\partial}{\partial r}\left(\rho_0 M_{0r}M_{0z}\phi_{am}\right)\\
\dsp +\frac{1}{\sqrt{1-M_\infty^2}}\frac{\partial}{\partial z}\left(\rho_0 ik_0 M_{0z}\phi_{am}\right)-\frac{ik_\infty M_\infty}{\left(1-M_\infty^2\right)^{\frac{3}{2}}}\frac{\partial}{\partial z}\left(\rho_0 M_{0z}^2\phi_{am}\right)\\
\dsp +\rho_0\left(-\frac{m^2}{r^2}-\frac{k_\infty^2 M_\infty^2}{\left(1-M_\infty^2\right)^2}\left(1-M_{0z}^2\right)
+\frac{ik_0 M_{0r}}{r}\right)\phi_{am}\\
\dsp +\rho_0\left( -ik_\infty\frac{M_{0r}M_{0z}M_\infty}{r\left(1-M_\infty^2\right)}+k_0^2-\frac{2k_\infty M_\infty k_0 M_{0z}}{\left(1-M_\infty^2\right)}\right)\phi_{am}=0\\
\end{array}\right.,
\EEQ
que l'on note par l'op\'erateur volumique $A_{Lm}\left(\phi_a\right)=0$.
\newline
$\bullet $ L'\'equation aux d\'eriv\'ees partielles (\ref{lorentztoutpartoutpartoupartou2r}) est elliptique.
\newline
$\bullet $ Les conditions aux limites du potentiel transform\'e s'\'ecrivent  :
\begin{enumerate} 
\item 
\BEQ{condtransparoirigide2r}
\left\{
\begin{array}{ll}
\dsp \frac{1}{\sqrt{1-M_\infty^2}}M_{0z} \ovra{n}.\overrightarrow{e_z}+M_{0r}\ovra{n}.\overrightarrow{e_r}=0&,\forall x\in \Gamma_R\\
\dsp \frac{\partial \phi_{am}}{\partial n_{A_{Lm}}}=0 &,\forall x\in \Gamma_R \\ 
\end{array}
\right.
\EEQ
\item 
\BEQ{condtransparoimodale2r}
\begin{array}{ll}
\dsp \frac{\partial\left( \phi_{am}-\phi_{am,inc}\right)}{\partial n_{A_{Lm}}}=T_{LMm}\left(\phi_{am}-\phi_{am,inc} \right) &,\forall x\in \Gamma_M \\
\end{array},
\EEQ
\end{enumerate}
o\`u $T_{LMm}$ d\'esigne l'op\'erateur Dirichlet-Neumann dans l'espace de Lorentz et s'obtient en appliquant la transformation de Lorentz \`a l'op\'erateur Dirichlet-Neumann (\ref{definitionoperateurmodalespacephysiquemodem}) permettant de borner le domaine modal du probl\`eme continu non transform\'e (\ref{probacoustiqueaxisymetriquecompletavecconditionsauxbordsrappelnumerodeuxreduitmodal}) :
\BEQ{definitionoperateurmodalespacephysique2r}
\left\lbrace
\begin{array}{c}
T_{LMm} :H^{\frac{1}{2}}\left(\Gamma_M\right)\rightarrow H^{-\frac{1}{2}}\left(\Gamma_M\right)\\
\phi_{am}\rightarrow \sum_{n\in\mathbb{N}}\mu_{mn}^-\left(\phi_a,\Xi_{rmn}\right)_{L^2\left(\Gamma_M\right)}\Xi_{rmn} \\
\end{array}
\right.,
\EEQ
o\`u l'on a not\'e :
\BEQ{constantedefinitionoperateurdirichletneumannespacelorentz2r}
\left\lbrace
\begin{array}{ll}
\dsp \gamma_{mn}^\pm=-k'_\infty M_\infty+\sqrt{1-M_\infty^2}\beta^\pm_{mn} &\text{ : les constantes de propagation axiales } \\
\dsp \mu_{mn}^\pm= i\rho_M\left(k_M M_M-\left(1-M_M^2\right)\beta^\pm_{mn}\right) &\text{ : les coefficients de l'op�rateur }T_{LMm} \\
\end{array}
\right.
\EEQ
 \newline
$\bullet $ L'\'equation (\ref{lorentztoutpartoutpartoupartou2r}) \'equivaut \`{a} la classique \'equation
de Helmholtz avec un simple d\'ephasage Doppler en dehors du domaine perturb\'e :
\BEQ{retourauondeaxisymetrique2r}
\dsp \frac{1}{r}\frac{\partial}{\partial r}\left(r \frac{\partial \phi_{am}}{\partial r}\right)-\frac{m^2}{r^2}\phi_{am}+\frac{\partial^2\phi_{am}}{\partial z^2}+\frac{k_\infty^2}{1-M_\infty^2}\phi_{am}=0
\EEQ
\newline
$\bullet $ Le th\'eor\`eme de Rellich implique l'unicit\'e du potentiel transform\'e v\'erifiant la condition de Sommerfeld :
\BEQ{conditiondeSommerfeld2r}
\lim_{R\rightarrow +\infty}\int_{S_R}\vert \frac{\partial \phi_{am}}{\partial n}-i\frac{k_\infty}{\sqrt{1-M_\infty^2}}\phi_{am}\vert^2 \ud \gamma =0,
\EEQ
o\`u $S_R$ (resp. $B_R$) d\'esigne la sph\`ere (resp. la boule) de rayon R et d'origine nulle avec $R$ choisi de telle sorte que $\Gamma_\infty\subset B_R$.
\newline
$\bullet $ Le probl\`eme acoustique global transform\'e rel\`eve de l'alternative de Fredholm et l'existence du potentiel acoustique transform\'e d\'ecoule de l'unicit\'e impliqu\'ee par la condition de Sommerfeld.
