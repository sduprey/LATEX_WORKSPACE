\subsection{Existence et unicit\'e}\label{unicitesommerfeldsubsection1}
\subsubsection{Probl\'{e}matique}\label{unicitesommerfeldsubsectionsub1} 
\begin{center}
\epsfig{file=lorentzavecheterogeneiteecoulement.eps,height=11.cm,width=14.50cm}
Transform\'ee de Lorentz et g\'eom\'etrie
\end{center}
\noindent Les domaines consid\'er\'es sont les domaines tridimensionnels et cart\'esiens $(x,y,z)$. N\'eanmoins, les d�monstrations d'existence et d'unicit� du probl�me acoustique complet sont r�dig�es dans un espace cart�sien bidimensionnel (l'axe $x$ correspond � l'axe de propagation $z$ et l'axe $y$ correspond aux axes  $(r,\theta)$) par souci de simplicit� �tant donn�e la taille des �quations \`a trois dimensions. Les d�monstrations sont totalement similaires dans un espace tridimensionnel et les principaux r\'esultats du paragraphe suivant sont explicit�s ult�rieurement dans le paragraphe (\ref{pareilenaxisubsectionsub1}) en coordonn�es cylindriques.
\newline La transform\'ee de Lorentz est appliqu\'ee \`a tout l'espace $\mathbb{R}^3=\overline{\Omega_i\cup\Omega_M\cup\Omega\cup\Omega_e}$. 
\newline Par abus de notation, les nouvelles variables physiques et les nouveaux domaines transform�s sont not�s comme les anciennes variables et anciens domaines.
\newline Ce paragraphe prouve l'existence et l'unicit\'e de la solution du probl\`eme acoustique complet.
\subsubsection{Unicit\'e et condition de rayonnement}\label{unicitesommerfeldsubsectionsub2}
\noindent On note $A_L \left(\phi_a\right)$ l'op\'erateur volumique d\'efini par l'\'equation aux d\'eriv\'ees partielles suivante  dans $\left(\overline{\Omega\cup\Omega_e}\right)\backslash\left(\Gamma_R\cup \Gamma_M\right)$ :
\BEQ{lorentzglobaldeuxdimensionsdeux}
\left\{\begin{array}{l}
\dsp
\frac{1}{1-M_\infty^2}\frac{\partial }{\partial
x}\left(\rho_0\frac{\partial \phi_a}{\partial
x}\right)+\frac{\partial }{\partial y}\left(\rho_0\frac{\partial
\phi_a}{\partial y}\right)\\
\dsp
-\frac{1}{1-M_\infty^2}\frac{\partial}{\partial x
}\left(\rho_0M_{0x}^2\frac{\partial
\phi_a}{\partial x}\right)
-\frac{1}{\sqrt{1-M_\infty^2}}\frac{\partial}{\partial x
}\left(\rho_0 M_{0x}M_{0y}\frac{\partial
\phi_a}{\partial y}\right)
\\
\dsp
-\frac{1}{\sqrt{1-M_\infty^2}}\frac{\partial}{\partial y
}\left(\rho_0 M_{0x}M_{0y}\frac{\partial
\phi_a}{\partial x}\right)-\frac{\partial}{\partial
y}\left(\rho_0 M_{0y}^2\frac{\partial
\phi_a}{\partial y}\right)
\\
\dsp
+\rho_0\left(\frac{ik_\infty
M_\infty}{\left(1-M_\infty^2\right)^{\frac{3}{2}}}\left(1-M_{0x}^2\right)+\frac{ik_0
M_{0x}
}{\sqrt{1-M_\infty^2}}\right)\frac{\partial\phi_a}{\partial
x}\\
\dsp
+\rho_0\left(ik_0M_{0y}-\frac{ik_\infty M_\infty
M_{0x}M_{0y}}{(1-M_\infty^2)}\right)\frac{\partial\phi_a}{\partial
y}+\frac{ik_\infty
M_\infty}{(1-M_\infty^2)^\frac{3}{2}}\frac{\partial }{\partial
x}\left(\rho_0\phi_a\right)\\
\dsp +\frac{1}{\sqrt{1-M_\infty^2}}\frac{\partial }{\partial
x}\left(\rho_0ik_0
M_{0x}\phi_a\right)-\frac{ik_\infty
M_\infty}{\left(1-M_\infty^2\right)^\frac{3}{2}}\frac{\partial
}{\partial x}\left(\rho_0M_{0x}^2\phi_a\right)
\\
\dsp
+\frac{\partial}{\partial y }\left(\rho_0ik_0M_{0y}
\phi_a\right)-\frac{ik_\infty
M_\infty}{1-M_\infty^2}\frac{\partial }{\partial
y}\left(\rho_0M_{0x}
M_{0y}\phi_a\right)
\\
\dsp +\rho_0\left(\frac{-k_\infty^2M_\infty^2}{\left(1-M_\infty^2\right)^2}
\left(1-M_{0x}^2\right)+k_0^2-\frac{2k_0
k_\infty M_{0x}M_\infty
}{\left(1-M_\infty^2\right)}\right)\phi_a=0,\\
\end{array}
\right.
\EEQ
La d�riv�e conormale de l'op�rateur elliptique $A_L$ est not\'ee $\dsp \frac{\partial}{\partial n_{A_L}}$.
\newline
$T_{LM}$ d\'esigne l'op\'erateur Dirichlet-Neumann transform\'e dans l'espace de Lorentz, qui s'obtient en appliquant la transformation de Lorentz \`a l'op\'erateur Dirichlet-Neumann (\ref{definitionoperateurmodalespacephysiquemodem}) permettant de borner le domaine modal du probl\`eme continu non transform\'e (\ref{probacoustiqueaxisymetriquecompletavecconditionsauxbordsrappelnumerodeuxreduitmodal}) :
\BEQ{definitionoperateurmodalespacephysique}
\left\lbrace
\begin{array}{c}
T_{LM} :H^{\frac{1}{2}}\left(\Gamma_M\right)\rightarrow H^{-\frac{1}{2}}\left(\Gamma_M\right)\\
\phi_{a}\rightarrow \sum_{(m,n)\in \mathbb{Z}\times\mathbb{N}}\mu_{mn}^-\langle\phi_a,\Xi_{rmn}\rangle_{L^2\left(\Gamma_M\right)}\Xi_{rmn} \\
\end{array}
\right.
\EEQ
Le caract\`ere incident ou r\'efl\'echi d'un mode dans l'espace de Lorentz d\'epend du caract\`ere de son image par la transform\'ee de Lorentz inverse :
\BEQ{constantedefinitionoperateurdirichletneumannespacelorentznumerodeux}
\left\lbrace
\begin{array}{ll}
\dsp \phi_{a,inc}=\sum_{(m,n)\in \mathbb{Z}\times\mathbb{N}} \langle\phi_{a,inc},\Xi_{rmn}\rangle_{L^2\left(\Gamma_M\right)} e^{i\gamma_{mn}^+ x}\Xi_{rmn}(y) &,\forall (x,y)\in \Gamma_M \\
\dsp \phi_a-\phi_{a,inc}=\sum_{(m,n)\in \mathbb{Z}\times\mathbb{N}} \langle \phi_a-\phi_{a,inc},\Xi_{rmn}\rangle_{L^2\left(\Gamma_M\right)} e^{i\gamma_{mn}^- x}\Xi_{rmn}(y) &,\forall (x,y)\in \Gamma_M \\
\end{array}
\right.,
\EEQ
o\`u l'on a not\'e :
\BEQ{constantedefinitionoperateurdirichletneumannespacelorentz}
\left\lbrace
\begin{array}{ll}
\dsp \gamma_{mn}^\pm=-k'_\infty M_\infty+\sqrt{1-M_\infty^2}\beta^\pm_{mn},&\text{ les constantes propagatives axiales }  \\
\dsp \mu_{mn}^\pm= i\rho_M\left(k_M M_M-\left(1-M_M^2\right)\beta^\pm_{mn}\right),&\text{ les coefficients de l'op�rateur }T_{LM} \\
\end{array}
\right. ,
\EEQ
avec $\dsp k_\infty'=\frac{k_\infty}{\sqrt{1-M_\infty^2}}$.
\newline
Si le Mach moteur \'egale le Mach infini, la transformation de Lorentz annule simplement l'\'ecoulement et l'on retombe sur le classique op\'erateur Dirichlet-Neumann d'un guide d'ondes. 
\newline
L'op\'erateur Dirichlet-Neumann est \'ecrit pour un guide d'ondes \`a trois dimensions par abus de notation : son expression bidimensionnelle est conceptuellement identique (seules les fonctions de Bessel radiales sont remplac\'ees par la classique base de Fourier $L^2$ en cosinus d'un segment. Les constantes propagatives axiales et les coefficients de l'op\'erateur $T_{LM}$ restent conceptuellement identiques.
\begin{theorem}\label{dupreypremier}
\text{\newline}\\
\noindent $\bullet $  Le potentiel acoustique transform\'e dans l'espace de Lorentz cart\'esien bidimensionnel $\dsp \left(\overline{\Omega\cup\Omega_e}\right)\backslash\left(\Gamma_R\cup \Gamma_M\right)$ v\'erifie l'\'equation aux d\'eriv\'ees partielles elliptique $A_L \left(\phi_a\right)=0$.
\text{\newline}\\
$\bullet $ Les conditions aux limites du potentiel transform\'e et de l'\'ecoulement transform\'e s'\'ecrivent :
\begin{enumerate} 
\item 
\BEQ{condtransparoirigide}
\left\{
\begin{array}{ll}
\dsp \frac{1}{\sqrt{1-M_\infty^2}}M_{0x} \ovra{n}.\overrightarrow{e_x}+M_{0y}\ovra{n}.\overrightarrow{e_y}=0, & \forall x\in \Gamma_R\\
\dsp \frac{\partial \phi_a}{\partial n_{A_L}}=0, &\forall x\in \Gamma_R \\ 
\end{array}
\right.
\EEQ
\item 
\BEQ{condtransparoimodale}
\begin{array}{ll}
\dsp \frac{\partial\left( \phi_a-\phi_{a,inc}\right)}{\partial n_{A_L}}=T_{LM}\left(\phi_a-\phi_{a,inc} \right), & \forall x\in \Gamma_M \\
\end{array},
\EEQ
\end{enumerate}
\text{\newline}\\
$\bullet $  L'\'equation aux d\'eriv\'ees partielles (\ref{lorentzglobaldeuxdimensionsdeux}) est \'equivalente \`a l'\'equation de Helmholtz, l\`a o\` u l'\'ecoulement est uniforme.
\BEQ{eqpottransinfondesbidim}
\frac{\partial^2 \phi_a}{\partial x^2}+\frac{\partial^2 \phi_a}{\partial y^2}+\left(k'_\infty\right)^2\phi_a=0,
\EEQ
\text{\newline}\\
$\bullet $ Le potentiel transform\'e v\'erifiant $\dsp 
\lim_{R\rightarrow +\infty}\int_{S_R}\vert \frac{\partial \phi_a}{\partial n} -ik'_\infty\phi_a\vert^2 \ud \gamma =0
$ est unique. Cette condition est la condition classique de Sommerfeld appliqu\'ee au potentiel acoustique transform\'e, o\`u $S_R$ (resp. $B_R$) d\'esigne la sph\`ere (resp. la boule) de rayon R et d'origine nulle avec $R$ choisi de telle sorte que $\Gamma_\infty\subset B_R$.
\end{theorem}
\noindent Preuve du th\'eor\`eme \ref{dupreypremier} :
\newline
Le diff\'eomorphisme $C^\infty$ de la transform\'ee de Lorentz avec les param\`{e}tres de l'\'ecoulement constant \`{a} l'infini (\ref{lorentzfrequent}) est appliqu\'e dans tout le domaine ext\'erieur $\mathbb{R}^3=\overline{\Omega_i\cup\Omega_M\cup\Omega\cup\Omega_e}$ \`a l'\'equation aux d\'eriv\'ees partielles cart\'esienne bidimensionnelle du potentiel acoustique (\ref{eqac}). Le changement d'inconnue conduit \`a l'\'equation aux d\'eriv\'ees partielles (\ref{lorentzglobaldeuxdimensionsdeux}) pour le nouveau potentiel acoustique. Les nouvelles conditions de bord s'obtiennent de la m\^eme fa\c con. La nullit\'e de la d\'eriv\'ee conormale dans l'espace physique est conserv\'ee par la transformation de Lorentz : (\ref{condtransparoirigide}). La transform\'ee de Lorentz appliqu\'ee \`a l'op\'erateur diff\'erentiel de bord modal (\ref{definitionoperateurmodalespacephysiquemodem}) donne l'�quation (\ref{condtransparoimodale}).
\newline
L'ellipticit\'e de l'op\'erateur $A_L$ s'obtient en �crivant la partie principale de l'op�rateur $A_L$ sous forme conservative :
\BEQ{ellipticitepotentiel}
 \dsp 
{\rm div}\left(A\left(x,y\right)\nabla\phi_a\left(x,y\right)\right)
\text{ avec }
A\left(x,y\right)=\rho_0\left(
\begin{array}{cc}
   \dsp    \frac{1-M_{0x}^2}{1-M_\infty^2}       &  \dsp  \frac{M_{0x}M_{0y}}{1-M_\infty^2}    \\
  \dsp    \frac{M_{0x}M_{0y}}{1-M_\infty^2}      &   \dsp  1-M_{0y}^2     \\
\end{array}\right)
\EEQ
%% \begin{displaymath}
%%  \dsp <A(x,y)X,X>=\rho_0\left(\begin{array}{cc} X_1 & X_2 \\ \end{array}\right)
%% \left(
%% \begin{array}{cc}
%%      \dsp  \frac{1-M_{0x}^2}{1-M_\infty^2}       &   \dsp \frac{M_{0x}M_{0y}}{1-M_\infty^2}    \\
%%    \dsp   \frac{M_{0x}M_{0y}}{1-M_\infty^2}      &   \dsp  1-M_{0y}^2     \\
%% \end{array}\right)\left(\begin{array}{c} X_1 \\ X_2 \\ \end{array}\right)
%% \end{displaymath}
\begin{displaymath}
 \dsp <A(x,y)X,X>=\rho_0
\left(\begin{array}{cc} \dsp  \frac{X_1}{\sqrt{1-M_\infty^2}} & X_2 \\ \end{array}\right)
\left(I-^t\overrightarrow{M}_0 \overrightarrow{M}_0\right)  
\left(\begin{array}{c}  \dsp \frac{X_1}{\sqrt{1-M_\infty^2}} \\ X_2 \\ \end{array}\right)
\end{displaymath}
%% \begin{displaymath}
%% <A(x,y)X,X>=\rho_0(\vert\left(\begin{array}{c} \dsp  \frac{X_1}{\sqrt{1-M_\infty^2}} \\ X_2 \\ \end{array}\right) \vert^2-\vert\overrightarrow{M}_0.\left(\begin{array}{c}  \dsp \frac{X_1}{\sqrt{1-M_\infty^2}} \\ X_2 \\ \end{array}\right)\vert^2
%% \end{displaymath}
\begin{displaymath}
\langle A(x,y)X,X\rangle\ge\rho_0\left(1-\vert\overrightarrow{M}_0\vert^2\right)\vert\left(\begin{array}{c} \frac{X_1}{\sqrt{1-M_\infty^2}} \\ X_2 \\ \end{array}\right) \vert^2
>C(\rho_0,M_0,\Omega)\vert\left(\begin{array}{c} X_1 \\ X_2 \\ \end{array}\right) \vert^2
\end{displaymath}
\text{L'in\'egalit\'e d'ellipticit\'e de l'\'equation aux d\'eriv\'ees partielles (\ref{lorentzglobaldeuxdimensionsdeux}) s'\'ecrit :\newline}
\BEQ{inegaliteelliptique}
\forall \left(x,y\right)\in\Omega, \forall X=\left(\begin{array}{c} X_1 \\ X_2 \\ \end{array}\right)\in \mathbb{R}^2\text{,  }
<A(\overrightarrow{x})X,X>\ge C(\rho_0,M_0,\Omega)\vert\left(\begin{array}{c} X_1 \\ X_2 \\ \end{array}\right) \vert^2
\EEQ
La preuve de (\ref{eqpottransinfondesbidim}) s'obtient en rempla�ant l'\'ecoulement variable $\rho_0$, $k_0$ et $\ovra{M_0}$ par ses valeurs \`a l'infini.
\newline On note $\dsp a_L(\ . \ , \ . \ )$ la forme sesquilin\'eaire volumique suivante :
\BEQ{definitionformbilinvolumlorglobpartout}
\left\{
\begin{array}{l}
\dsp a_L(\phi_a,\psi)=\int_{\Omega}\rho_0\left( \frac{1-M_{0x}^2}{(1-M_\infty)^2}  \right)\frac{\partial\phi_a}{\partial
x}\frac{\partial \overline{\psi}}{\partial
x}
+\int_{\Omega}\rho_0\left(1-M_{0y}^2\right)\frac{\partial
\phi_a}{\partial y}\frac{\partial
\overline{\psi}}{\partial y} 
\\
\dsp -\frac{1}{\sqrt{1-M_\infty^2}}\int_{\Omega}\rho_0
M_{0x}M_{0y}\frac{\partial \phi_a}{\partial y
}\frac{\partial\overline{\psi}}{\partial
x}
-\frac{1}{\sqrt{1-M_\infty^2}}\int_{\Omega}\rho_0
M_{0x}M_{0y}\frac{\partial \phi_a}{\partial x
}\frac{\partial\overline{\psi}}{\partial
y}\\
\dsp -\int_{\Omega}\rho_0(\frac{ik_\infty
M_\infty}{(1-M_\infty^2)^{\frac{3}{2}}}(1-M_{0x}^2)+\frac{ik_0
M_{0x}
}{\sqrt{1-M_\infty^2}})\frac{\partial\phi_a}{\partial
x}\overline{\psi}\\
\dsp -\int_{\Omega}\rho_0(ik_0
M_{0y}-\frac{ik_\infty M_\infty
M_{0x}M_{0y}}{1-M_\infty^2})\frac{\partial\phi_a}{\partial
y}\overline{\psi}
\\
\dsp +\int_{\Omega}\rho_0\left( \frac{ik_\infty
M_\infty (1-M_{0x}^2)}{(1-M_\infty^2)^\frac{3}{2}}+ \frac{i k_0
M_{0x}}{\sqrt{1-M_\infty^2}}\right)\phi_a\frac{\partial\overline{\psi}}{\partial
x}
\\
\dsp
+\int_{\Omega}\rho_0\left(i k_0M_{0y}-\frac{ik_\infty M_\infty M_{0x}M_{0y}}{1-M_\infty^2}\right)\phi_a\frac{\partial
\overline{\psi}}{\partial y }
\\
\dsp -\int_{\Omega} \rho_0(\frac{-k_\infty^2 M_\infty^2}{(1-M_\infty^2)^2}(1-M_{0x}^2)+k_0^2-\frac{2k_0 k_\infty M_{0x}M_\infty}{1-M_\infty^2})\phi_a\overline{\psi}\\
\end{array}
\right. 
\EEQ
On note $\dsp b_L(\ . \ , \ . \ )$ la forme sesquilin\'eaire de bord suivante :
\BEQ{defintdebordlorglob}
\left\{
\begin{array}{l}
%\dsp\langle T_{LM}\left(\phi_a\right),\psi\rangle_{L^2\left(\Gamma_M\right)}+\int_{\Gamma_\infty} \ovra{\nabla}\phi_a .\ovra{n} \overline{\psi}= \\
\dsp
 b_L(\phi_a , \psi ) =\int_{\partial\Omega}\rho_0(\frac{1-M_{0x}^2}{1-M_\infty^2})\frac{\partial\phi_a}{\partial
x}\overline{\psi}\overrightarrow{n}.\overrightarrow{e_x}-\int_{\partial\Omega}\rho_0\frac{M_{0x}M_{0y}}{\sqrt{1-M_\infty^2}}\frac{\partial\phi_a}{\partial
y}\overline{\psi}\overrightarrow{n}.\overrightarrow{e_x}
\\
\dsp
+\int_{\partial\Omega}\rho_0(1-M_{0y}^2)\frac{\partial\phi_a}{\partial
y}\overline{\psi}\overrightarrow{n}.\overrightarrow{e_y}
-\int_{\partial\Omega}\rho_0\frac{M_{0x}M_{0y}}{\sqrt{1-M_\infty^2}}\frac{\partial\phi_a}{\partial
x}\overline{\psi}\overrightarrow{n}.\overrightarrow{e_y}\\
\dsp +\int_{\partial\Omega}\rho_0(\frac{ik_\infty
M_\infty}{(1-M_\infty^2)^\frac{3}{2}}+\frac{ik_0
M_{0x}}{\sqrt{1-M_\infty^2}}-\frac{ik_\infty M_\infty
M_{0x}^2}{(1-M_\infty^2)^\frac{3}{2}})\phi_a\overline{\psi}\overrightarrow{n}.\overrightarrow{e_x}\\
\dsp +
\int_{\partial\Omega}\rho_0(ik_0
M_{0y}-\frac{ik_\infty M_\infty
M_{0x}M_{0y}}{1-M_\infty^2})\phi_a\overline{\psi}\overrightarrow{n}.\overrightarrow{e_y}\\
\end{array}
\right.
\EEQ
Par lin\'earit\'e, montrer l'unicit\'e du probl\`eme \'equivaut \`a montrer que pour une source sonore incidente nulle, la condition de Sommerfeld implique une unique solution nulle.
\begin{lemma}\label{dupreylemme}
On suppose donc la source sonore incidente nulle : $\phi_{a,inc}=0$. Le probl\`eme transform\'e (\ref{lorentzglobaldeuxdimensionsdeux}),
(\ref{condtransparoirigide}) et (\ref{condtransparoimodale}) se formule variationnellement dans $\Omega$ :
\text{\newline}
\BEQ{formvarbidimlorentz}
\left\{
\begin{array}{l}
\dsp\text{ Trouver }  \phi_a \in H^1(\Omega)  \text{ tel que : }\\
\dsp a_L(\phi_a,\psi)=\langle T_{LM}\left(\phi_a\right),\psi\rangle_{L^2\left(\Gamma_M\right)}+\int_{\Gamma_\infty} \ovra{\nabla}\phi_a .\ovra{n} \ \overline{\psi}\\
   \forall \psi \in H^1(\Omega)\\
\end{array}
\right.,
\EEQ 
\newline
$\bullet $
La solution du probl\`eme transform\'e soumise � la condition de Sommerfelfd v\'erifie le bilan d'�nergie :
\BEQ{egaliteenergie}
\begin{array}{c}
\lim_{R\rightarrow +\infty}k'_\infty\vert\vert \phi_a \vert\vert_{L^2(S_R)}\\
\dsp + \\
\dsp\sum_{(m,n)\in\mathbb{Z}\times\mathbb{N}}\sum_\pm \Im m\left(\mu_{mn}^\pm\right)\vert\langle\phi_a,\Xi_{rmn}\rangle_{L^2\left(\Gamma_M\right)}\vert^2\vert\vert \Xi_{rmn}\vert\vert^2_{L^2\left(\Gamma_M\right)}\\
\dsp =0\\
\end{array},
\EEQ o� la somme modale ci-dessus ne se fait que sur les modes propagatifs.
\newline
$\bullet $
En l'absence de modes incidents propagatifs \`a l'entr\'ee de la nacelle ($\phi_{a,inc}=0$) :
\begin{itemize}
\item \BEQ{decroissancealinfini}
\lim_{R\rightarrow +\infty}\vert\vert \phi_a \vert\vert_{L^2(S_R)}=0
\EEQ
\item \BEQ{decroissancealinfininumerodeux}
\lim_{R\rightarrow +\infty}\vert\vert \frac{\partial \phi_a}{\partial n} \vert\vert_{L^2(S_R)}=0
\EEQ
\end{itemize}
\end{lemma}
\noindent Preuve du lemme \ref{dupreylemme} :
\newline
La formulation variationnelle (\ref{formvarbidimlorentz}) s'obtient � partir de l'\'equation aux d\'eriv\'ees partielles elliptique (\ref{lorentzglobaldeuxdimensionsdeux}) que l'on a multipli�e par une fonction test et int�gr�e par parties. L'expression des conditions aux limites (\ref{condtransparoirigide}) et (\ref{condtransparoimodale}) dans l'int\'egrale de bord (\ref{defintdebordlorglob}) donne le second membre. $a_L(\ . \ , \ . \ )$ est la forme sesquilin\'eaire volumique associ�e � l'op�rateur $A_L$ et $\dsp b_L(\phi_a , \psi )=\langle T_{LM}\left(\phi_a\right),\psi\rangle_{L^2\left(\Gamma_M\right)}+\int_{\Gamma_\infty} \ovra{\nabla}\phi_a .\ovra{n} \overline{\psi}$ est la forme sesquilin\'eaire de bord associ\'ee \`a la d\'eriv\'ee conormale $\dsp \frac{\partial}{\partial n_{A_L}}$.
\newline
La forme sesquilin\'eaire volumique (\ref{definitionformbilinvolumlorglobpartout}) s'\'ecrit sous la forme synth\'etique :
\BEQ{formblinlorentzglobmemefonctest}
\begin{array}{c}
\dsp a_L\left(\phi_a,\phi_a\right)=\int_{\Omega}f_1(x,y)\vert \phi_a \vert^2 +\int_{\Omega} \langle A(x,y)\nabla \phi_a,\overline{\nabla \phi_a} \rangle \\
\dsp +\int_{\Omega}i f_2(x,y)\frac{\partial \phi_a}{\partial x} \overline{\phi_a}
\dsp +\overline{\int_{\Omega}i f_2(x,y)\frac{\partial \phi_a}{\partial x} \overline{\phi_a}} \\ 
\dsp +\int_{\Omega}i f_3(x,y)\frac{\partial \phi_a}{\partial y} \overline{\phi_a}
\dsp +\overline{\int_{\Omega}i f_3(x,y)\frac{\partial \phi_a}{\partial y} \overline{\phi_a}} \\
\end{array},
\EEQ
o\`u $f_1$, $f_2$ et $f_3$ sont trois fonctions $W^{1,\ \infty}\left(\Omega\right)$ \`a valeurs r\'eelles.
La matrice $A(x,y)$ d\'efinie par (\ref{ellipticitepotentiel}) \'etant r\'eelle, sym\'etrique, d\'efinie et positive, la forme sesquilin\'eaire volumique v\'erifie :
\BEQ{lemmenecessaireegaliteenergie}
\Im m\left(a_L\left(\phi_a,\phi_a\right)\right)=0,\ \ \forall \phi_a \in H^1\left(\Omega\right)
\EEQ
L'application de $\phi_a$ comme fonction-test dans la formulation variationnelle (\ref{formvarbidimlorentz}) donne l'\'egalit\'e :
\begin{displaymath}
a_L(\phi_a,\phi_a)=\int_{\Gamma_\infty}\frac{\partial \phi_a}{\partial n}\overline{\phi_a} 
+\langle T_{LM}\left(\phi_a\right),\phi_a\rangle_{L^2\left(\Gamma_M\right)}
\end{displaymath}
\BEQ{premiereegaliteenergie}
a_L(\phi_a,\phi_a)=\int_{\Gamma_\infty}(\frac{\partial \phi_a}{\partial n}-ik'_\infty \phi_a)\overline{\phi_a} +ik'_\infty\int_{\Gamma_\infty}\phi_a\overline{\phi_a} +\langle T_{LM}\left(\phi_a \right),\phi_a\rangle_{L^2\left(\Gamma_M\right)}
\EEQ
En utilisant la d\'ecomposition de $\phi_a$ sur les modes de conduit et l'orthogonalit\'e de ceux-ci :
\begin{displaymath}
\langle T_{LM}\left(\phi_a \right),\phi_a\rangle_{L^2\left(\Gamma_M\right)}=\sum_{(m,n)\in\mathbb{Z}\times\mathbb{N}}\sum_\pm \mu_{mn}^\pm\vert\langle\phi_a,\Xi_{rmn}\rangle_{L^2\left(\Gamma_M\right)}\vert^2\vert\vert \Xi_{rmn}\vert\vert^2_{L^2\left(\Gamma_M\right)}
\end{displaymath}
En identifiant partie r\'eelle et imaginaire dans l'\'egalit\'e (\ref{premiereegaliteenergie}) :
\begin{displaymath}
k'_\infty\vert\vert \phi_a\vert\vert_{L^2(\Gamma_\infty)}^2+\sum_{(m,n)\in\mathbb{Z}\times\mathbb{N}}\sum_\pm \Im m\left(\mu_{mn}^\pm\right)\vert\langle\phi_a,\Xi_{rmn}\rangle_{L^2\left(\Gamma_M\right)}\vert^2\vert\vert \Xi_{rmn}\vert\vert^2_{L^2\left(\Gamma_M\right)}
\end{displaymath}
\begin{displaymath}
=-\Im m\left(\int_{\Gamma_\infty}(\frac{\partial \phi_a}{\partial n}-ik'_\infty \phi_a)\overline{\phi_a} \right)
\le\vert\vert \phi_a\vert\vert_{L^2(\Gamma_\infty)}\vert\vert\frac{\partial \phi_a}{\partial n}-ik'_\infty \phi_a \vert\vert_{L^2(\Gamma_\infty)}
\end{displaymath}
La somme modale ne se fait que sur les modes propagatifs :
\BEQ{caracterisationmodespropagatifs}
\Im m\left(\mu_{mn}^\pm\right)=0 \Longleftrightarrow \text{ le mode est \'evanescent}
\EEQ
\BEQ{caracterisationmodespropatifsincidents}
\Im m\left(\mu_{mn}^\pm\right)<0 \Longleftrightarrow \text{ le mode est propagatif incident}
\EEQ
La condition de Sommerfeld conclut (\ref{egaliteenergie}). En l'absence de modes propagatifs incidents, on a une somme de deux termes positifs nulle, il d\'ecoule (\ref{decroissancealinfini}) et la nullit� des coefficients des modes propagatifs r�fl�chis :
\BEQ{coefficientsmodauxnuls}
\vert a_{mn}^-\vert=0, \ \ \forall n\in \mathbb{N}\text{ tel que } \Xi^-_{mn}\text{ soit un mode propagatif}
\EEQ
La condition de Sommerfeld et (\ref{decroissancealinfini}) implique (\ref{decroissancealinfininumerodeux}) :
\begin{displaymath}
\vert\vert \frac{\partial\phi_a}{\partial n}\vert\vert_{L^2(\Gamma_\infty)}\le k'_\infty\vert\vert \phi_a\vert\vert_{L^2(\Gamma_\infty)}+ \vert\vert \frac{\partial \phi_a}{\partial n}-ik'_\infty \phi_a\vert\vert_{L^2(\Gamma_\infty)}
\end{displaymath}
\newline Le th\'eor\`eme de Rellich \cite{FR} et le principe de continuation unique pour une \'equation aux d\'eriv\'ees partielles elliptique (la r\'egularit\'e $W^{1,\ \infty}\left(\Omega\right)$ des coefficients variables provenant de l'\'ecoulement porteur est suffisante) concluent la nullit\'e de $\phi_a$ partout et donc la nullit\'e de tous les coefficients r\'efl\'echis (pas seulement les propagatifs).
\newline Citons le th\'eor\`eme de Rellich \cite{FR} :
\begin{theorem}[Rellich] Soit $\phi_a$ une fonction de classe $C^2$ pour $\vert x \vert $ assez grand, et v\'erifiant l'\'equation de Helmholtz : $\Delta \phi_a+ \lambda \phi_a=0$ pour $\vert x\vert\ge r_0$ (avec $\lambda>0$).
\newline
Alors seuls deux cas de figures sont possibles :
\begin{itemize}
\item \BEQ{conditionnulleinfiniRellich} \phi_a(x)=0\text{ pour }\vert x\vert>r_0 \EEQ
\item \BEQ{conditionnonnulleinfiniRellich} \forall R_0>r_0, \exists  C>0 \text{ tel que } \int_{R_0<\vert x\vert <R} \vert \phi_a(x)\vert^2\ud x\ge CR, \text{ pour R assez grand}\EEQ
\end{itemize}
\end{theorem}
\noindent
Le th\'eor\`eme de Rellich et (\ref{decroissancealinfini}) entra\^inent (\ref{conditionnulleinfiniRellich}). Etant donn� $\epsilon>0$, il existe d'apr\`es (\ref{decroissancealinfini}) un r\'eel $R_\epsilon>0$ tel que $\vert\vert \phi_a \vert\vert^2_{L^2\left(S_{r}\right)}\le \epsilon$, $\forall r\ge R_\epsilon$.
\begin{displaymath}
\forall R>R_\epsilon, \int_{R_\epsilon <\vert x\vert <R} \vert \phi_a\vert^2 \ {\rm d}x=\int_{R_\epsilon<r<R}\ud r\int_{S_r}\vert \phi_a\vert^2 r \ud\theta \le \epsilon \left(R-R_\epsilon\right)
\end{displaymath}
Le cas (\ref{conditionnonnulleinfiniRellich}) du th\'eor\`eme de Rellich n'est pas possible : $\phi_a=0$ pour $\vert x\vert>r_0$. Le principe de prolongement unique d'une �quation aux d�riv�es partielles strictement elliptique � coefficients variables de r�gularit� $W^{1,\ \infty}\left(\Omega\right)$ (qui est la r�gularit� de l'�coulement porteur dont on peut extraire une sous-suite convergente pour $W^{1,\ \infty}\left(\Omega\right)$ d'apr\`es le th\'eor\`eme d'Ascoli) implique la nullit\'e de $\phi_a$ partout dans $\overline{\Omega_M\cup\Omega\cup\Omega_e}\backslash\Gamma_R$. Par unicit\'e de la d\'ecomposition modale dans le guide d'ondes, les coefficients r\'efl\'echis \'evanescents sont nuls. Par lin\'earit\'e, la solution du probl\`eme transform\'e est unique.

